% Options for packages loaded elsewhere
\PassOptionsToPackage{unicode}{hyperref}
\PassOptionsToPackage{hyphens}{url}
%
\documentclass[
]{article}
\usepackage{amsmath,amssymb}
\usepackage{lmodern}
\usepackage{ifxetex,ifluatex}
\ifnum 0\ifxetex 1\fi\ifluatex 1\fi=0 % if pdftex
  \usepackage[T1]{fontenc}
  \usepackage[utf8]{inputenc}
  \usepackage{textcomp} % provide euro and other symbols
\else % if luatex or xetex
  \usepackage{unicode-math}
  \defaultfontfeatures{Scale=MatchLowercase}
  \defaultfontfeatures[\rmfamily]{Ligatures=TeX,Scale=1}
\fi
% Use upquote if available, for straight quotes in verbatim environments
\IfFileExists{upquote.sty}{\usepackage{upquote}}{}
\IfFileExists{microtype.sty}{% use microtype if available
  \usepackage[]{microtype}
  \UseMicrotypeSet[protrusion]{basicmath} % disable protrusion for tt fonts
}{}
\makeatletter
\@ifundefined{KOMAClassName}{% if non-KOMA class
  \IfFileExists{parskip.sty}{%
    \usepackage{parskip}
  }{% else
    \setlength{\parindent}{0pt}
    \setlength{\parskip}{6pt plus 2pt minus 1pt}}
}{% if KOMA class
  \KOMAoptions{parskip=half}}
\makeatother
\usepackage{xcolor}
\IfFileExists{xurl.sty}{\usepackage{xurl}}{} % add URL line breaks if available
\IfFileExists{bookmark.sty}{\usepackage{bookmark}}{\usepackage{hyperref}}
\hypersetup{
  pdftitle={Simulación},
  pdfauthor={Uriel Paluch},
  hidelinks,
  pdfcreator={LaTeX via pandoc}}
\urlstyle{same} % disable monospaced font for URLs
\usepackage[margin=1in]{geometry}
\usepackage{color}
\usepackage{fancyvrb}
\newcommand{\VerbBar}{|}
\newcommand{\VERB}{\Verb[commandchars=\\\{\}]}
\DefineVerbatimEnvironment{Highlighting}{Verbatim}{commandchars=\\\{\}}
% Add ',fontsize=\small' for more characters per line
\usepackage{framed}
\definecolor{shadecolor}{RGB}{248,248,248}
\newenvironment{Shaded}{\begin{snugshade}}{\end{snugshade}}
\newcommand{\AlertTok}[1]{\textcolor[rgb]{0.94,0.16,0.16}{#1}}
\newcommand{\AnnotationTok}[1]{\textcolor[rgb]{0.56,0.35,0.01}{\textbf{\textit{#1}}}}
\newcommand{\AttributeTok}[1]{\textcolor[rgb]{0.77,0.63,0.00}{#1}}
\newcommand{\BaseNTok}[1]{\textcolor[rgb]{0.00,0.00,0.81}{#1}}
\newcommand{\BuiltInTok}[1]{#1}
\newcommand{\CharTok}[1]{\textcolor[rgb]{0.31,0.60,0.02}{#1}}
\newcommand{\CommentTok}[1]{\textcolor[rgb]{0.56,0.35,0.01}{\textit{#1}}}
\newcommand{\CommentVarTok}[1]{\textcolor[rgb]{0.56,0.35,0.01}{\textbf{\textit{#1}}}}
\newcommand{\ConstantTok}[1]{\textcolor[rgb]{0.00,0.00,0.00}{#1}}
\newcommand{\ControlFlowTok}[1]{\textcolor[rgb]{0.13,0.29,0.53}{\textbf{#1}}}
\newcommand{\DataTypeTok}[1]{\textcolor[rgb]{0.13,0.29,0.53}{#1}}
\newcommand{\DecValTok}[1]{\textcolor[rgb]{0.00,0.00,0.81}{#1}}
\newcommand{\DocumentationTok}[1]{\textcolor[rgb]{0.56,0.35,0.01}{\textbf{\textit{#1}}}}
\newcommand{\ErrorTok}[1]{\textcolor[rgb]{0.64,0.00,0.00}{\textbf{#1}}}
\newcommand{\ExtensionTok}[1]{#1}
\newcommand{\FloatTok}[1]{\textcolor[rgb]{0.00,0.00,0.81}{#1}}
\newcommand{\FunctionTok}[1]{\textcolor[rgb]{0.00,0.00,0.00}{#1}}
\newcommand{\ImportTok}[1]{#1}
\newcommand{\InformationTok}[1]{\textcolor[rgb]{0.56,0.35,0.01}{\textbf{\textit{#1}}}}
\newcommand{\KeywordTok}[1]{\textcolor[rgb]{0.13,0.29,0.53}{\textbf{#1}}}
\newcommand{\NormalTok}[1]{#1}
\newcommand{\OperatorTok}[1]{\textcolor[rgb]{0.81,0.36,0.00}{\textbf{#1}}}
\newcommand{\OtherTok}[1]{\textcolor[rgb]{0.56,0.35,0.01}{#1}}
\newcommand{\PreprocessorTok}[1]{\textcolor[rgb]{0.56,0.35,0.01}{\textit{#1}}}
\newcommand{\RegionMarkerTok}[1]{#1}
\newcommand{\SpecialCharTok}[1]{\textcolor[rgb]{0.00,0.00,0.00}{#1}}
\newcommand{\SpecialStringTok}[1]{\textcolor[rgb]{0.31,0.60,0.02}{#1}}
\newcommand{\StringTok}[1]{\textcolor[rgb]{0.31,0.60,0.02}{#1}}
\newcommand{\VariableTok}[1]{\textcolor[rgb]{0.00,0.00,0.00}{#1}}
\newcommand{\VerbatimStringTok}[1]{\textcolor[rgb]{0.31,0.60,0.02}{#1}}
\newcommand{\WarningTok}[1]{\textcolor[rgb]{0.56,0.35,0.01}{\textbf{\textit{#1}}}}
\usepackage{graphicx}
\makeatletter
\def\maxwidth{\ifdim\Gin@nat@width>\linewidth\linewidth\else\Gin@nat@width\fi}
\def\maxheight{\ifdim\Gin@nat@height>\textheight\textheight\else\Gin@nat@height\fi}
\makeatother
% Scale images if necessary, so that they will not overflow the page
% margins by default, and it is still possible to overwrite the defaults
% using explicit options in \includegraphics[width, height, ...]{}
\setkeys{Gin}{width=\maxwidth,height=\maxheight,keepaspectratio}
% Set default figure placement to htbp
\makeatletter
\def\fps@figure{htbp}
\makeatother
\setlength{\emergencystretch}{3em} % prevent overfull lines
\providecommand{\tightlist}{%
  \setlength{\itemsep}{0pt}\setlength{\parskip}{0pt}}
\setcounter{secnumdepth}{-\maxdimen} % remove section numbering
\ifluatex
  \usepackage{selnolig}  % disable illegal ligatures
\fi

\title{Simulación}
\author{Uriel Paluch}
\date{21/11/2021}

\begin{document}
\maketitle

\hypertarget{metodos}{%
\subsection{Metodos}\label{metodos}}

\begin{Shaded}
\begin{Highlighting}[]
\NormalTok{integracion }\OtherTok{\textless{}{-}} \ControlFlowTok{function}\NormalTok{(funcion, limiteSuperior, limiteInferior, n)\{}
\NormalTok{  uniforme }\OtherTok{\textless{}{-}}\NormalTok{ limiteInferior }\SpecialCharTok{+}\NormalTok{ (limiteSuperior }\SpecialCharTok{{-}}\NormalTok{ limiteInferior) }\SpecialCharTok{*} \FunctionTok{runif}\NormalTok{(}\AttributeTok{n =}\NormalTok{ n)}
  
\NormalTok{  alturaPromedio }\OtherTok{\textless{}{-}} \DecValTok{1}\SpecialCharTok{/}\NormalTok{n }\SpecialCharTok{*} \FunctionTok{sum}\NormalTok{(}\FunctionTok{funcion}\NormalTok{(uniforme))}
  
\NormalTok{  anchoBase }\OtherTok{\textless{}{-}}\NormalTok{ limiteSuperior }\SpecialCharTok{{-}}\NormalTok{ limiteInferior}
  
\NormalTok{  desvioEstandar }\OtherTok{\textless{}{-}} \FunctionTok{sqrt}\NormalTok{( }\DecValTok{1}\SpecialCharTok{/}\NormalTok{(n}\DecValTok{{-}1}\NormalTok{) }\SpecialCharTok{*} \FunctionTok{sum}\NormalTok{((}\FunctionTok{funcion}\NormalTok{(uniforme) }\SpecialCharTok{*}\NormalTok{ (limiteSuperior }\SpecialCharTok{{-}}\NormalTok{ limiteInferior) }\SpecialCharTok{{-}}\NormalTok{ alturaPromedio }\SpecialCharTok{*}\NormalTok{ anchoBase)}\SpecialCharTok{\^{}}\DecValTok{2}\NormalTok{) )}
  
\NormalTok{  error }\OtherTok{\textless{}{-}}\NormalTok{ desvioEstandar}\SpecialCharTok{/}\FunctionTok{sqrt}\NormalTok{(n)}
  
\NormalTok{  resultados }\OtherTok{\textless{}{-}} \FunctionTok{list}\NormalTok{(}\StringTok{"error"} \OtherTok{=}\NormalTok{ error, }\StringTok{"alfa"} \OtherTok{=}\NormalTok{  alturaPromedio }\SpecialCharTok{*}\NormalTok{ anchoBase)}
  \FunctionTok{return}\NormalTok{(resultados)}
\NormalTok{\}}

\NormalTok{funcion }\OtherTok{\textless{}{-}} \ControlFlowTok{function}\NormalTok{(x)\{}
  \FunctionTok{return}\NormalTok{(}\FunctionTok{sqrt}\NormalTok{(x}\SpecialCharTok{+}\DecValTok{5}\NormalTok{)}\SpecialCharTok{*}\FunctionTok{sin}\NormalTok{(x))}
\NormalTok{\}}

\CommentTok{\# Fijo el seed para que me den los mismos resultados que la guia}
\FunctionTok{set.seed}\NormalTok{(}\DecValTok{123}\NormalTok{)}

\FunctionTok{integracion}\NormalTok{(}\AttributeTok{funcion =}\NormalTok{ funcion, }\AttributeTok{limiteSuperior =} \DecValTok{6}\NormalTok{, }\AttributeTok{limiteInferior =} \DecValTok{2}\NormalTok{, }\AttributeTok{n =} \DecValTok{10000}\NormalTok{)}
\end{Highlighting}
\end{Shaded}

\begin{verbatim}
## $error
## [1] 0.07177432
## 
## $alfa
## [1] -4.52639
\end{verbatim}

\hypertarget{integrales-como-esperaza}{%
\subsection{Integrales como esperaza}\label{integrales-como-esperaza}}

Supongamos que se quiere calcular la siguiente integral

\begin{equation*}
  \alpha = \int_{a}^{b} f(x)
\end{equation*}

Supongamos que contamos con un mecanismo para generar puntos
independientes y con distribución uniforme en el intervalo \([0;1]\). Si
se evalúa la función en cada uno de estos puntos y se promedian los
resultados, obtenemos una estimación de la integral mediante Monte
Carlo:

\textbf{Aclaración:} Esto es solo válido para el intervalo \([0;1]\)

\begin{equation*}
  \hat{\alpha} = \dfrac{1}{n} \sum_{i = 1}^{n} f(U_i)
\end{equation*}

Esto se debe a la Ley de los Grandes Números

\hypertarget{desvuxedo-estandar}{%
\subsubsection{Desvío estandar:}\label{desvuxedo-estandar}}

\begin{equation*}
  s_f = \sqrt{\dfrac{1}{n-1} \sum_{i = 1}^{n} (f(U_i) -\hat{\alpha}_n)^2}
\end{equation*}

\hypertarget{error}{%
\subsubsection{Error:}\label{error}}

\begin{equation*}
  \dfrac{s_f}{\sqrt(n)}
\end{equation*}

\hypertarget{quuxe9-sucede-cuando-el-intervalo-no-es-01}{%
\subsubsection{\texorpdfstring{¿Qué sucede cuando el intervalo no es
\([0;1]\)?}{¿Qué sucede cuando el intervalo no es {[}0;1{]}?}}\label{quuxe9-sucede-cuando-el-intervalo-no-es-01}}

Debo multiplicar los números aleatorios por el tamaño de la base (límite
superior - límite inferior), de esta forma los números estaran
contenidos siempre dentro del intervalo.

\begin{equation*}
  uniforme <- limiteInferior + (limiteSuperior - limiteInferior) * runif(100)
\end{equation*}

El valor obtenido lo múltiplico por el tamaño de la base (límiteSuperior
- límiteInferior) y eso me genera un área promedio que aproxima bien a
la integral.

Se puede transformar cualquier variable aleatoria en una uniforme y
viceversa, una uniforme en cualquier variable aleatoria. Esto se hace
con \(F^{-1}(X)\)

\hypertarget{movimiento-geomuxe9trico-browniano}{%
\subsection{Movimiento geométrico
Browniano}\label{movimiento-geomuxe9trico-browniano}}

Es el mas utilizado para simular el precio de una acción.
\begin{equation*}
  P_t = P_0 * e^{(\mu - 0.5 * \sigma^2) * T + \sigma * \sqrt{T}*\epsilon}
\end{equation*} Donde:\\
- \(\mu\) es el rendimiento esperado\\
- \(\sigma\) es la volantilidad\\
- \(P_0\) es el precio del activo en el momento cero\\
- \(P_T\) es el precio del activo en el momento T\\
- \(\epsilon\) es una variable aleatoria normal estandar\\
- T es el plazo

Si deseo un camino de precios puedo utilizar: \begin{equation*}
  P_{t+\Delta t} = P_t * e^{(\mu - 0.5 * \sigma^2) * \Delta t + \sigma * \sqrt{\Delta t}*\epsilon}
\end{equation*} donde \(\Delta t\) es una variación pequeña del tiempo

\hypertarget{procesos-multivariados}{%
\subsection{Procesos multivariados}\label{procesos-multivariados}}

En este caso consideramos que los retornos de los activos estan
relacionados entre si.

Movimiento Geométrico Browniano para precios de activos financieros:
\begin{equation*}
  P_{k,T} = P_{k.0} * e^{(\mu_k - 0.5 * \sigma_k^2) * T + \sigma_k * \sqrt{T}*\epsilon_k}
\end{equation*} Consideramos que los \epsilon\_k están correlacionados
entre si. Por lo tanto necesito la correlación como dato.

\end{document}
